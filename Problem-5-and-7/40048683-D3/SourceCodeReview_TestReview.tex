\documentclass[12pt]{article}
\usepackage[utf8]{inputenc}
\usepackage{fullpage}

\title{Deliverable - 3}
\author{Charanpreet Singh Bedi, 40048683}
\date{}
\begin{document}
\maketitle

\section*{Source code review for Function-9}
Following is the source code review from me for the function-9 ($x^y$), the code is reviewed on the basis of design patter, coding practices, non-functional requirements and structure of the code overall.
\begin{itemize}
    \item \textbf{Design pattern}: So the developer has used MVC design pattern to design the code for function-9, which I feel is a very viable option. It separates the view, model and controller of the system very well which allows the maintainability of the code much more easier. The code can also be easily be modularized, so overall this is a beneficial option. 
    \item \textbf{Coding Practices:} I looked over the code and saw that proper commenting has been done using JavaDoc, which is appreciated. The method names are camel-case and JFrame is used for extending the view of the code. One thing that is good about the code is, the use of getter and setter methods to store data. Also, no extra white spaces are found. The logs have not been mainatained, which would have been really helpful to figure out errors and also help during debugging.
    \item \textbf{Non functional requirements:} I compiled the code and saw that it is easily compiled and also easily debugged. The average response time to attain a result is less than 2 seconds, which is great too. Code is testable through the unit test cases provided by the developer. 
\end{itemize}
\subsection*{Conclusion}
Overall the code is well written and very well structured. The only flaw I could find was the missing logs.

\newpage

\section*{Test review for Function-10}
Following is the test review for the function-10 ($\sigma$);
\begin{itemize}
    \item All the methods for the function-10 are tested and they return positive response. The test cases cover edge values as well and also the test cases are run for decimal values with positive response. Test cases are fast and precise. 
    \item There is one thing I would like to mention here is that the unit test cases should be a part of the test suite which is a good practice.
\end{itemize}

\subsection*{DVCS}
Following is the address of the GitHub,\newline https://github.com/charan121317/soen-6011-project

\begin{thebibliography}{}

\bibitem{wikipedia_b}
Vogella: Unit Testing,
\\\texttt{https://www.vogella.com/tutorials/JUnit/article.html}

\end{thebibliography}
\end{document}
 