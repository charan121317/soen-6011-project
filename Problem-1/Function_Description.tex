\documentclass[10pt]{article}
\usepackage[utf8]{inputenc}
\usepackage{fullpage}

\title{Problem 1: Description about Beta function}
\author{Charanpreet Singh Bedi, 40048683}
\date{}
\begin{document}
\maketitle


\section*{Introduction}
Beta function is also known as Euler's integral of first kind and is very important in calculus and is basically an association between input and output values. The beta function is used to determine average time to complete some tasks in the time related problems. Beta function has a very close connection to the gamma function which also the generalisation of the factorial function. The beta function is defined as follows: 


    $$\beta(x,y) =\int_{0}^{1} t^{x-1} (1-t)^{y-1} dt$$
    for Re (x) $>$ 0 and Re (y) $>$ 0
    
\subsubsection*{Domain}
Re (x) $>$ 0 and Re (y) $>$ 0

\subsubsection*{Co-domain}
all real numbers

\subsubsection*{Characteristics}
\begin{itemize}
    \item The beta function is uniquely defined for positive numbers and complex numbers with positive real parts. It is approximated for other numbers.
    \item Beta function is symmetric,$$ \beta(x,y) = \beta(y,x)$$
    \item A key property of beta function is that it is closly related to the gamma function
    $$\beta(x,y) = \frac{\gamma(x)\gamma(y)}{\gamma(x+y)}$$
\end{itemize}

\subsubsection*{Applications}
\begin{itemize}
    \item The Beta function was the first known 'Scattering' amplitude in String theory.
    \item In physics and string theory the beta function is used to calculate and reproduce scattering amplitudes in terms of the Regge trajectories.
\end{itemize}

\begin{thebibliography}{}
\bibitem{Riddhi D}
Beta function and its Applications.
\textit{by Riddhi D.}
Department of Physics and Astronomy
The University of Tennessee
Knoxville, TN 37919, USA
\bibitem{wikipedia}
Wikiedia: Beta Function,
\\\texttt{https://en.wikipedia.org/wiki/Beta\_function}
\bibitem{brilliant}
Brilliant: The Beta Function,
\\\texttt{https://brilliant.org/wiki/beta-function/}
\bibitem{quora}
Quora: What is a Beta Function,
\\\texttt{https://www.quora.com/What-is-beta-function}
\end{thebibliography}

\end{document}
 